\documentstyle[%
righttag%
,11pt%
,amssymb%
,russian%               remove in Eng.
,izvru%                 Set izven% in Eng.
]{amsart}

% lef@uic.nnov.ru
% NB: \text{...} constructions
% seek \wt and repl. 4\Tilde
\newcommand{\e}{\varepsilon}
\newcommand{\de}{\delta}
\newcommand{\subs}{\subset}
\newcommand{\Om}{\Omega}
\newcommand{\ze}{\zeta}
\newcommand{\Per}{\operatorname{Per}}
\newcommand{\Orb}{\operatorname{Orb}}
\newcommand{\pr}{\operatorname{pr}}
\newcommand{\Ls}{\operatornamewithlimits{Ls}}
\newcommand{\Lim}{\operatornamewithlimits{Lim}}
\newcommand{\tts}[1]{}

\theoremstyle{definition}
\theorembodyfont{\rm}
\newtheorem{examnonum}{\hskip\parindent Пример}
\renewcommand{\theexamnonum}{}

%\hoffset=-15mm%                  For Eng.
\setcounter{page}{19}
\god{2006}
\nomer{10~(533)} %                        Remove in Eng.
%\nomer{Vol.\,50, No.\,10}%               Set in Eng.
%\str{\pageref{begin}--\pageref{end}}%  Set in Eng.
\udk{517.987}

\author{\it Л.С.\,Ефремова}
% NB:   ^^ remove in Eng.
\title{О неблуждающем          %%%%%% nonwandering ??
множестве и центре
некоторых косых произведений
отображений интервала}
\thanks{Работа выполнена при финансовой поддержке Российского фонда
фундаментальных исследований (код проекта 04-01-00457).}

\date{}
%\pagestyle{myheadings}%         Set in Eng.
\pagestyle{plain} %              Remove in Eng.
\begin{document}
%\thispagestyle{plain}%          Set in Eng.
\maketitle
\label{begin}




Исследования систем дифференциальных уравнений с цилиндрическим
фазовым пространством [1], аттрактора Лоренца [2],
описание моделей вполне развитой турбулентности [3]
приводят к динамическим системам со структурой косого
произведения.

Изучению неблуждающего множества и центра
непрерывного косого
произведения отображений интервала с замкнутым множеством
периодических точек в базе посвящена работа [4].
В [5] показано, что основополагающую роль в исследовании
динамики косого произведения отображений интервала
играют специальные многозначные функции.
В [6] с использованием многозначных функций, введенных в [5] и [6],
начато исследование неблуждающего множества и
центра $C^1$-гладкого косого произведения отображений
интервала, фактор-отображения и отображения, в слоях которого
удовлетворяются условия типа $\Om$-устойчивости.

Естественное продолжение статей [5] и [6] привело к
новым доказательствам результатов [4],
публикуемым в данной работе.
Укажем, что доказательство теоремы о глубине центра непрерывного
косого произведения отображений интервала в статье [4]
отсутствует. Доказательства основаны на использовании
оригинальных многозначных функций, графики которых в
фазовом пространстве совпадают с
рассмотренными в статье основными предельными множествами
динамической системы.


\section{Предварительные сведения}

Рассмотрим косое произведение отображений интервала
\begin{equation}\label{e:1}
F(x, y)=(f(x), g_x(y)) \ \ \text{для всех} \ \ (x, y)\in I,
\end{equation}
где $I=I_1\times I_2$ --- замкнутый прямоугольник на плоскости
($I_1$, $I_2$ --- замкнутые промежутки).
Обозначим через $T^0(I)$ ($T^1(I)$) пространство всех непрерывных
($C^1$-гладких)
отображений вида (\ref{e:1}) с $C^0$-нормой ($C^1$-нормой).
В силу (\ref{e:1}) для каждого
натурального $n$ и произвольной точки $(x, y)$
справедливо равенство
\begin{equation}\label{e:2}
F^n(x, y)=(f^n(x), g_{x, n}(y)), \ \ \text{где} 
\ \ g_{x, n}=g_{f^{n-1}(x)}\circ\dotsb\circ g_{f(x)}\circ g_x.
\end{equation}
Обозначим через $\wt g_x$ отображение $g_{x, n}$, если
$x$ --- периодическая точка $f$ ($x\in \Per(f)$), а
$n$ --- (наименьший) период $x$.

В статье приведено новое доказательство
следующего утверждения из [4].


\begin{thm}
Пусть фактор-отображение $f$
косого произведения отображений интервала $F\in T^0(I)$
имеет замкнутое множество $\Per(f)$.
Тогда неблуждающее множество $\Om(F)$ отображения $F$
представимо в виде
$$
\Om(F)=\ov{\bigcup\limits_{x\in \Per(f)}
\{x\}\times\Om(\wt g_x)},
$$
где $\Om(\wt g_x)$ --- неблуждающее множество отображения $\wt g_x$.
\end{thm}

Данная работа содержит доказательство следующей теоремы,
которая только лишь
сформулирована, но не доказана в [4].


\begin{thm}
Пусть отображение $F\in T^0(I)$
удовлетворяет условиям теоремы~$1$.
Тогда центр $C(F)$ отображения $F$
представим в следующем виде\/${:}$
$$
C(F)=\Om(F_{|\Om(F)})=\ov{\bigcup\limits_{x\in \Per(f)}
\{x\}\times\Om(\wt g_{x|\Om(\Tilde g_x)})},
$$
где $F_{|\Om(F)}$ $(\wt g_{x|\Om(\Tilde g_x)})$ --- сужение
отображения $F$ $(\wt g_x)$ на его неблуждающее множество
$\Om(F)$ $(\Om(\wt g_x))$.
\end{thm}

Доказательства сформулированных результатов основаны на
использовании многозначных функций из [5] и [6].
Пусть $2^{I_2}$ есть пространство всех замкнутых подмножеств
отрезка $I_2$ с экспоненциальной топологией.


\begin{defn}
$\Om$-функцией отображения $F\in T^0(I)$ называется функция
$\ze:\Om(f)\to 2^{I_2}$ такая, что при любом $x\in\Om(f)$
выполнено равенство $\ze(x)=\Om(F)(x)$, где
$\Om(F)(x)=\{y\in I_2:(x, y)\in\Om(F)\}$ --- срез
неблуждающего множества $\Om(F)$ по $x$.

$C$-функцией отображения $F\in T^0(I)$ называется функция
$\ze^*: C(f)\to 2^{I_2}$ такая, что при любом $x\in C(f)$
выполнено равенство $\ze^*(x)= C(F)(x)$, где
$C(F)(x)=\{y\in I_2:(x, y)\in C(F)\}$ --- срез
центра $C(F)$ по $x$.
\end{defn}

Будем использовать также вспомогательные многозначные функции
$$
\eta_n:\Om(f^n)\to 2^{I_2} \ \  \text{и}  
\ \ \eta^*_n:C(f^n)=C(f)\to 2^{I_2},
$$
полагая
при всех $n\geq 1$
$\eta_n(x)=\Om(g_{x, n})$, каково бы ни было
$x\in\Om(f^n)$, и 
$\eta^*_n(x)=C(g_{x, n})$, каково бы ни было $x\in C(f)$
соответственно.

Приведем утверждения о свойствах непрерывных
отображений отрезка (пространство таких отображений с
$C^0$-нормой обозначим $C^0(I_1)$),
использующиеся в доказательствах теорем 1 и 2.


\begin{prop}
Если множество периодических точек $\Per(f)$ отображения
$f\in C^0(I_1)$ замкнуто, то
\begin{itemize}
\item[(a)] $T(f)=\{1, 2, 2^2,\dotsc, 2^{\nu}\}$, где
$T(f)$ --- множество периодов периодических точек $f$, $0\leq\nu\leq
+\infty$   $[7],$
\item[(b)] для каждой точки $x^0\in \Per(f)$ найдется
окрестность $U_1(x^0)\subs I_1$ со свойством
$$
U_1(x^0)\bigcap f^n(U_1(x^0))\ne\emptyset
$$
в том и только том случае, если $n$ кратно периоду
$n(x^0)= 2^{i_0}$ точки $x^0$   $[8]$.
\end{itemize}
\end{prop}


\begin{prop}
Для отображения
$f\in C^0(I_1)$ следующие утверждения эквивалентны:
\begin{itemize}
\item[(a)] множество $\Per(f)$ периодических точек $f$
замкнуто,
\item[(b)] справедливы равенства $\Per(f)=\Om(f)=C(f)$, где $C(f)$ ---
центр отображения $f$,
\item[(c)] $\omega$-предельное множество $\omega_f(x)$ произвольной
точки $x$ --- периодическая орбита  $[7]$,~$[9]$.
\end{itemize}
\end{prop}


\begin{prop}[{\series{m}\shape{n}\selectfont [9]}]
Для отображения $f\in C^0(I_1)$ справедливо равенство
$$
C(f)=\ov {\Per(f)}=\Omega(f_{|\Omega(f)}).
$$
\end{prop}


\begin{prop}
Если множество периодических точек отображения
$f\in C^0(I_1)$ замкнуто, а последовательность периодических точек $f$
сходится к $x^0$, то последовательность траекторий этих точек
сходится к траектории точки $x^0$  $[10]$.
\end{prop}

Будем использовать далее некоторые вспомогательные утверждения о
свойствах отображений из $T^0(I)$, имеющиеся в [4].


\begin{lem}
Пусть периодические точки фактор-отображения $f:I_1\to I_1$
отображения $F\in T^0(I)$ образуют замкнутое множество.
Тогда, если $x^0$ --- периодическая точка $f$ периода
$n(x^0)= 2^{i_0}$ $(i_0\geq 1)$, то следующие утверждения
эквивалентны\/{:}

$(\mathrm a)$ $(x^0, y^0)\in\Om(F)$,\quad
$(\mathrm b)$ $(x^0, y^0)\in\Om(F^{n(x^0)})$.
\end{lem}


\begin{lem}
Если $F\in T^0(I)$, то
$(\mathrm a)$  $\Om(f)=\pr\Om(F)$,
$(\mathrm b)$  $C(f)=\pr(C(F))$, здесь $\pr(\cdot)$ ---
проекция множества на ось абсцисс.
\end{lem}

В силу леммы 2 и определения 1 неблуждающее множество
$\Om(F)$ и центр $C(F)$ отображения $F\in T^0(I)$
служат графиками многозначных $\Om$-функции и $C$-функции
соответственно
в фазовом пространстве $I$ динамической системы (\ref{e:1}).


\section{Доказательство теоремы 1. Пример}


Доказательство теоремы 1 разобьем на ряд шагов, рассмотренных
в леммах 3 и 4. Будем использовать $\Om$-функцию и
вспомогательные функции $\eta_n$.
Укажем, что в силу предложения~2 отображения
$\eta_n$ определены на множестве $\Per(f)$.

Пусть $\Per(f, n)$ означает множество
периодических точек
фактор-отображения $f$, периоды которых делят $n$.


\begin{lem}
Если множество $Per(f)$
фактор-отображения $f$ отображения
$F\in T^0(I)$ замкнуто,
то верно равенство
\begin{equation}\label{e:Ls}
\Ls\limits_{n\to +\infty}\eta_{n|\Per(f, n)}=
\ov{\bigcup\limits_{x\in \Per(f)}\{x\}\times\Om(\wt g_x),}
\end{equation}
где
$\{{\eta_n}_{|\Per(f, n)}\}_{n\geq 1}$ ---
последовательность графиков функций
${\eta_n}_{|\Per(f, n)}$ в $I$, $\Ls(\cdot)$ ---
верхний предел последовательности множеств.
\end{lem}


\begin{pf}
Пусть $z_0(x_0, y_0)\in\Ls\limits_{n\to +\infty}
\eta_{n|\Per(f, n)}$.
Это эквивалентно существованию последовательности точек
$z_{n_i}(x_{n_i}, y_{n_i})\in\eta_{n_i|\Per(f, n_i)}$
такой, что
$$
z_0=\lim\limits_{i\to +\infty} z_{n_i},
$$
где $n_1 < n_2 <\dotsb < n_i\ldots$ 
([11], гл.\,2, п.\,29, III).
Из определения графика функции $\eta_{n_i}$ следует, что
$z_{n_i}(x_{n_i}, y_{n_i})\in\eta_{n_i|\Per(f, n_i)}$
в том и только том случае, если
$x_{n_i}\in \Per(f, n_i)$, $n_i=2^{\nu_i}p_i$
(здесь $2^{\nu_i}$ --- период $x_{n_i}$), и
$y_{n_i}\in\Om(\wt g_{x_{n_i}}^{p_i})$ $(i\geq 1)$.
Так как
$\Om(\wt g_{x_{n_i}}^{p_i})\subseteq
\Om(\wt g_{x_{n_i}})$ (как показано в [12], равенство
$\Om(\wt g_{x_{n_i}}^{p_i})=
\Om(\wt g_{x_{n_i}})$ для произвольного непрерывного отображения
$\wt g_{x_{n_i}}$ может не выполняться),
то получаем $z_0(x_0, y_0)\in
\ov{\bigcup\limits_{x\in \Per(f)}\{x\}\times\Om(\wt g_x)}$.
Таким образом,
$\Ls\limits_{n\to +\infty}\eta_{n|
\Per(f)}\subset \ov{\bigcup\limits_{x\in \Per(f)}\{x\}\times\Om(\wt g_x)}$.

Обратно, пусть $z_0(x_0, y_0)\in
\ov{\bigcup\limits_{x\in \Per(f)}\{x\}\times\Om(\wt g_x)}$.
Тогда найдется последовательность точек
$\{z_n(x_n, y_n)\}_{n\geq 1}\subset
\bigcup\limits_{x\in \Per(f)}\{x\}\times\Om(\wt g_x)$
такая, что
$\lim\limits_{n\to +\infty}z_n=z_0$. $\phantom{\int\limits^{1^1}_{1_1}}$

Так как $\Per(f)$ --- замкнутое множество, то возможны следующие случаи:
\begin{itemize}
\item[(i)] последовательность наименьших периодов точек
$x_n$ финально постоянна;
\item[(ii)] существует подпоследовательность
$\{x_{n_i}\}_{i\geq 1}\subset\{x_n\}_{n\geq 1}$,
состоящая из периодических точек $f$, наименьшие
периоды которых $2^{\nu_i}$ образуют строго возрастающую последовательность.
\end{itemize}

В первом случае положим $n_i=2^{\nu_0}(2(i+ n_0)-1)$, здесь
$2^{\nu_0}$ --- период произвольной точки $x_n$ при всех
$n\geq n_0$ для некоторого $n_0\geq 1$,
во втором ---
$n_i=2^{\nu_i}$, $i\geq 1$. С использованием [12] получаем, что 
$\eta_{n_i| \Per(f, n_i)}(x_{n_i})=\Om(\wt g_{x_{n_i}})$ при любом $i\geq 1$.
Следовательно,
$z_0(x_0, y_0)\in\Ls\limits_{n\to +\infty}
\eta_{n| \Per(f, n)}$, и справедливо противоположное включение
$\ov{\bigcup\limits_{x\in \Per(f)}\{x\}\times\Om(\wt g_x)}
\subset\Ls\limits_{n\to +\infty}\eta_{n| \Per(f, n)}$.
Равенство (\ref{e:Ls}) установлено.
\end{pf}


\begin{lem}
Пусть выполнены условия леммы $3$. Тогда
$$
\ze=\Ls\limits_{n\to +\infty}{\eta_n}_{| \Per(f, n)},
$$
где ${\eta_n}_{| \Per(f, n)}$ $(n\geq 1)$, $\ze$ --- графики
соответствующих функций в $I$.
\end{lem}


\begin{pf}
В силу равенства (\ref{e:Ls}) верно включение
\begin{equation}\label{e:4.5}
\Ls\limits_{n\to +\infty}
{\eta_n}_{| \Per(f, n)}\subseteq\ze.
\end{equation}
Установим справедливость противоположного включения
\begin{equation}\label{e:4.6}
\ze\subseteq\Ls\limits_{n\to +\infty}
{\eta_n}_{| \Per(f, n)}.
\end{equation}
Будем использовать множество $\pi=\Per(f)\times I_2$.

Пусть $\pi^*=\bigcap\limits_{n=0}^{+\infty}F^n(\pi)$.
Тогда $F: \pi^*\to \pi^*$ --- сюръекция.

1. Убедимся сначала, что
\begin{equation}\label{e:4.7}
\ze_{F_{| \pi}}=\Ls\limits_{n\to +\infty}{\eta_n}_{|
\Per(f, n)}.
\end{equation}
Равенство (\ref{e:4.7}) справедливо, если
$$
\Ls\limits_{n\to
+\infty}{\eta_n}_{| \Per(f, n)} =\lim_{n\to
+\infty}F^n(\pi)= \bigcap\limits_{n=0}^{+\infty}F^n(\pi)=\pi^*.
$$
Поэтому будем предполагать далее, что
\begin{equation}\label{e:4.7^*}
\Ls\limits_{n\to +\infty}{\eta_n}_{|\Per(f, n)} \ne \pi^*.
\end{equation}
Возьмем произвольно точку
\begin{equation}\label{e:4.7^{**}}
z^0(x^0, y^0)\in
\pi^*\setminus\Ls\limits_ {n\to +\infty}{\eta_n}_{| \Per(f,n)}.
\end{equation}
Покажем, что $z^0\notin\ze_{F_{| \pi}}$. Действительно, пусть
в $I$  \ $\e$-окрестность $U_{\e}(z^0)$ точки $z^0$ и $\e$-окрест\-ность 
$U_{\e}(\Ls\limits_{n\to +\infty}{\eta_n}_{|\Per(f, n)})$ множества
$\Ls\limits_{n\to +\infty}{\eta_n}_{|\Per(f, n)}$ выбраны так, что
\begin{equation}\label{e:4.8}
U_{\e}(z^0)\bigcap
U_{\e}(\Ls\limits_{n\to +\infty}{\eta_n}_{|\Per(f, n)})=\emptyset.
\end{equation}
Обозначим через $n_0=2^{\nu_0}$ период точки $x^0\in \Per(f)$.
Используя равномерную непрерывность отображения $g_{x, n_0}$,
по числу $\frac{\e}{2}> 0$ укажем $\de> 0$ так, что
для любых $x, x'\in I_1$, $y, y'\in I_2$, удовлетворяющих
неравенствам $|x - x'|<\de$, $|y - y'|<\de$, выполнено неравенство
\begin{equation}\label{e:4.9}
|g_{x, n_0}(y) - g_{x', n_0}(y')|<\frac{\e}{2}.
\end{equation}
Из утверждения (b) предложения 1 следует, что найдется интервал 
$U_1(x^0)\subseteq (x^0 -\de, x^0 +\de)$, в котором для
всех $x\in U_1(x^0)\bigcap \Per(f)$ справедливо
\begin{equation}\label{e:4.10}
f^{n_0j}(x)\in (x^0 -\de, x^0 +\de),\quad  j\geq 0.
\end{equation}

Выделим два случая: (a) и (b).

(a). Предположим, что существует $\de\geq\frac{\e}{2}$,
для которого имеет место (\ref{e:4.9}).
Тогда в силу (\ref{e:4.9}) и (\ref{e:4.10}) для любой точки $(x, y)$, где
$x\in U_1(x^0)\bigcap \Per(f)$, $|y - y^0|<\frac{\e}{2}$, и любого
$j\geq 1$ справедливо
\begin{equation}\label{e:4.11}
|g_{x, n_0j}(y) - g_{x^0, n_0j}(y^0)|=
|g_{x, n_0j}(y) - \wt g_{x^0}^j(y^0)|<\frac{\e}{2}.
\end{equation}
Так как при некотором натуральном
$j_0=j_0(U_{2, \frac{\e}{2}} (\Omega(\wt g_{x^0})))$,
не зависящем от $y^0$,
выполнено неравенство
$$
\sum\limits_{j=0}^{+\infty}\chi_{A_1}(\wt g_{x^0}^j(y^0))\leq j_0
$$
(здесь $U_{2, \frac{\e}{2}}(\Omega(\wt g_{x^0}))$
--- $\frac{\e}{2}$-окрестность множества неблуждающих точек
$\Omega(\wt g_{x^0})$ в $I_2$, $\chi_{A_1}$ ---
характеристическая функция множества
$A_1=I_2\setminus U_{2, \frac{\e}{2}}(\Omega(\wt g_{x^0}))$),
то из (\ref{e:2}) и (\ref{e:4.11}) следует
$\sum\limits_{j=0}^{+\infty}\chi_{A_2}(g_{x, n_0j}(y))\leq j_0$,
где $\chi_{A_2}$ --- характеристическая функция
множества
$$
A_2=\pi^*\setminus U_{\e}
(\Ls\limits_{n\to +\infty}{\eta_n}_{|\Per(f, n)}).
$$
Это вместе с (\ref{e:4.8}) влечет
$z^0\notin\Omega({F^{n_0}}_{|\pi})$.
Используя лемму 1, получаем отсюда $z^0\notin\ze_{F_{|\pi}}$.

(b). Рассмотрим случай, когда неравенство (\ref{e:4.9})
выполнено лишь для $\de<\frac{\e}{2}$.
Возьмем произвольно и зафиксируем отрицательную полутраекторию
точки $y^0$ относительно отображения
$\wt g_{x^0}=g_{x^0, n_0}$. Так как $z^0\in \pi^*$, а
$F: \pi^*\to \pi^*$ --- сюръекция, то отрицательная полутраектория
точки $y^0$ относительно $\wt g_{x^0}$ корректно определена.
Пусть точки $\{y^{-j}\}_{j\geq 0}$ образуют выделенную отрицательную
полутраекторию точки $y^0$, здесь
$g_{x^0, n_0j}(y^{-j})=\wt g_{x^0}^j(y^{-j})=y^0$.
Тогда для любой точки $(x, y)\in \pi^*$ и любого $j\geq 1$, где
$x\in U_1(x^0)\bigcap \Per(f)$, $|y - y^{-j}|<\de$,
верно неравенство
\begin{equation}\label{e:4.12}
|g_{x, n_0j}(y) - g_{x^0, n_0j}(y^{-j})|=
|g_{x, n_0j}(y) - \wt g_{x^0}^j(y^{-j})|<\frac{\e}{2}.
\end{equation}
Так как при
некотором натуральном $j_1=j_1(U_{2, \de}(\Omega(\wt g_{x^0})))$,
не зависящем от $y^0$,
имеет место неравенство
$$
\sum\limits_{j=0}^{+\infty}\chi_{A_3}(y^{-j})\leq j_1
$$
(здесь $U_{2, \de} (\Omega(\wt g_{x^0}))$
--- $\de$-окрестность множества неблуждающих точек
$\Omega(\wt g_{x^0})$ в $I_2$,
$\chi_{A_3}$ ---
характеристическая функция множества
$A_3=I_2\setminus U_{2, \de}(\Omega(\wt g_{x^0}))$),
то из (\ref{e:4.12}) следует, что найдется натуральное
число $j_2\geq j_1$ такое, что при всех $j\geq j_2$
для $j$-го полного прообраза в $\pi^*$ относительно отображения
$F_{| \pi^*}^{n_0}$  \  $\frac{\e}{2}$-окрестности
$U_{\frac{\e}{2}}^{\pi^*}(z^0)=U_{\frac{\e}{2}}(z^0)\bigcap\pi^*$
в $\pi^*$ точки $z^0$ справедливо включение
$$
(F_{| \pi^*}^{n_0})^{-j}
(U_{\frac{\e}{2}}^{\pi^*}(z^0))\subset\pi^*\bigcap U_{\e}
(\Ls\limits_{n\to +\infty}{\eta_n}_{|\Per(f, n)}).
$$
Отсюда, используя (\ref{e:4.8}), получаем
$U_{\frac{\e}{2}}^{\pi^*}(z^0)\bigcap (F_{| \pi^*}^{n_0})^{-j}
(U_{\frac{\e}{2}}^{\pi^*}(z^0))=\emptyset$
при любом $j\geq j_2$. Следовательно,
$U_{\frac{\e}{2}}^{\pi^*}(z^0)\bigcap F^{n_0j}
(U_{\frac{\e}{2}}^{\pi^*}(z^0))=\emptyset$
каково бы ни было $j\geq j_2$.
Последнее вместе с леммой 1 влечет за собой
$z^0\notin\ze_{F_{| \pi}}$.
Так как для $z^0\in\pi\setminus\pi^*$ выполнено
$z^0\notin\ze_{F_{| \pi}}$, то
равенство (\ref{e:4.7}) доказано.

2. Завершим доказательство леммы 4.

Пусть $I^*=\bigcap\limits_{n=0}^{+\infty}F^n(I)$.
Тогда $F$ сюръективно на $I^*$, и $\ze\subset I^*$.
Так как для любой точки $(x, y)\in I^*$ выполнено включение
$\omega_F((x, y))\subset(\Ls\limits_{n\to +\infty}
{\eta_n}_{| \Per(f, n)})$
(где $\omega_F((x, y))$ -- $\omega$-предельное множество
$F$-траектории точки $(x, y)$),
то в силу утверждения (c) предложения 2 лемма 4
справедлива, если
$\pi^*=\Ls\limits_{n\to
+\infty}{\eta_n}_{| \Per(f, n)}$.

Пусть выполнено неравенство (\ref{e:4.7^*}). Возьмем произвольно
и зафиксируем точку $z^0$ так, чтобы выполнялось (\ref{e:4.7^{**}}).
Покажем, что $z^0\notin\ze$.

Пусть $\e$-окрестности $U_{\e}(z^0)$ и
$U_{\e}(\Ls\limits_{n\to +\infty}{\eta_n}_{|\Per(f, n)})$ соответственно
точки $z^0 $ и множества
$\Ls\limits_{n\to +\infty}{\eta_n}_{|
\Per(f, n)}$ выбраны в силу (\ref{e:4.8}).
Используя результаты п.\,1 и теорему Дж.\,Биркгофа [13],
для окрестности
$U_{\frac{\e}{2}}(\Ls\limits_{n\to +\infty}{\eta_n}_{|
\Per(f, n)})$ множества
$\Ls\limits_{n\to +\infty}{\eta_n}_{|
\Per(f, n)}$ укажем натуральное число
$n_1=n_1(U_{\frac{\e}{2}}(\Ls\limits_{n\to
+\infty}{\eta_n}_{| \Per(f, n)}))$ так, чтобы
выполнялось неравенство
\begin{equation}\label{e:4.13}
\sum\limits_{n=0}^{+\infty}\chi_{A_4}
(F^n(x^0, y^0))\leq n_1,
\end{equation}
где $\chi_{A_4}$ --- характеристическая функция
множества
$A_4=I\setminus U_{\frac{\e}{2}}(\Ls\limits_{n\to
+\infty}{\eta_n}_{| \Per(f, n)})$.
Воспользуемся равномерной непрерывностью отображения $F$ и
по числу $\frac{\e}{2}> 0$ укажем $\de'> 0$ так, чтобы
для любых $x, x'\in I_1$, $y, y'\in I_2$, удовлетворяющих
неравенствам $|x - x'|<\de'$, $|y - y'|<\de'$, выполнялось
\begin{equation}\label{e:4.14}
F(x', y')\in U_{\frac{\e}{2}}(F(x, y)),
\end{equation}
где $U_{\frac{\e}{2}}(F(x, y))$ --- $\frac{\e}{2}$-окрестность
точки $F(x, y)$ в $I$.

Предположим, что существует $\de'\geq\frac{\e}{2}$, для которого
выполнено (\ref{e:4.14}). Тогда положительная полутраектория
точки $z^0$  \  $\{F^n(z^0)\}_{n\geq 0}$ ``отслеживается''
положительной полутраекторией $\{F^n(x, y)\}_{n\geq 0}$
произвольной точки $(x, y)\in U_{\de'}(z^0)\bigcap I^*$
($U_{\de'}(z^0)$ --- $\de'$-окрестность $z^0$ в $I$)
с точностью до $\frac{\e}{2}$.
Используя неравенство (\ref{e:4.13}), получаем отсюда
$\sum\limits_{n=0}^{+\infty}\chi_{A_5}(F^n(x^0, y^0))\leq n_1$,
где $\chi_{A_5}$ --- характеристическая функция множества
$A_5=I\setminus U_{\e}(\Ls\limits_{n\to
+\infty}{\eta_n}_{| \Per(f, n)})$.
Последнее вместе с предложением 2 влечет за собой
$z^0\notin\ze$.

Пусть (\ref{e:4.14}) имеет место лишь для
$\de'< \frac{\e}{2}$. Сюръективность $F_{| I^*}$ позволяет
рассматривать отрицательные полутраектории (относительно
отображения $F_{| I^*}$) точек множества $I^*$.
Проводя аналогичные рассуждения для
отрицательных полутраекторий точки $z^0$ и точек
$(x, y)\in U_{\frac{\e}{2}}(z^0)\bigcap I^*$, также
убеждаемся в том, что
$z^0\notin\ze$.
Таким образом, включение (\ref{e:4.6}) установлено.
Справедливость леммы 4 следует из
(\ref{e:4.5}) и (\ref{e:4.6}).
\end{pf}

Леммы 3 и 4 влекут за собой справедливость теоремы 1.$\quad\square$
\medskip


Обратим внимание на то, что существенным элементом
доказательства леммы 4 (а, следовательно,
и теоремы 1) является возможность отслеживать
положительные или выделенные отрицательные полутраектории
точек в слоях над периодическими точками
фактор-отображения соответственно положительными или
некоторыми отрицательными полутраекториями ``близких''
точек в аналогичных слоях (отметим, что классическая теорема о семействах
$\e$-траекторий [14] не применима в рассматриваемом случае).
Важную роль в нахождении таких полутраекторий играет
свойство непрерывных отображений отрезка
с замкнутым множеством периодических точек, указанное в предложении 4.

Из теоремы 2 следует, что множество
$$
\Omega_w(F)=\Omega(F)\setminus
\Big(\bigcup\limits_{x\in \Per(f)}\{x\}\times\Omega(\wt g_x)\Big)
$$
является граничным в $\Omega(F)$ и, если оно не пусто,
состоит из точек $(x, y)$, каждая из которых блуждает
относительно содержащего ее слоя $\{x\}\times I_2$.

Приведем пример $C^1$-гладкого отображения
$F\in T^1([0, 1]^2)$ с замкнутым множеством
периодических точек в базе и непустым множеством $\Omega_w(F)$.

Потребуются прямоугольники
$$
\Pi_1=(0, 1]\times \bigg(\frac{1}{8}, \frac{1}{8} + \frac{x}{16}\bigg), \quad
\Pi_2=[0, 1]\times \bigg(\frac{1}{8} +\frac{x}{16}, \frac{3}{16} +
\frac{x}{32}\bigg),
$$
а также их верхняя и нижняя границы соответственно
\begin{alignat*}{2}
\partial_l(\Pi_1)&=[0, 1]\times\bigg\{\frac{1}{8}\bigg\}, &\quad
\partial_u(\Pi_1)&=[0, 1]\times\bigg\{\frac{1}{8}+\frac{x}{16}\bigg\};\\
%
\partial_l(\Pi_2)&=\partial_u(\Pi_1), &\quad
\partial_u(\Pi_2)&=[0, 1]\times\bigg\{\frac{3}{16}+\frac{x}{32}\bigg\}.
\end{alignat*}
Будем использовать следующие строго монотонные по $y$ функции:
\begin{align*}
\lambda_x(y)-\frac14&=
\begin{cases}
\frac{x}{4}\exp \left[-\frac{ \exp (-(\frac{1}{8} -y)^{-2})}
{(\frac{1}{8} +\frac{x}{16}- y)^2}\right],
& \text{если} \ (x, y)\in\Pi_1; \\
\frac{x}{4}, & \text{если} \ (x, y)\in\partial_l(\Pi_1);\\
0, & \mbox{если} \ (x, y)\in\partial_u(\Pi_1),
\end{cases}
\\
%
\mu_x(y)-\frac14&=
\begin{cases}
-A\exp \left[-\frac{\exp (-{(\frac{3}{16}+
\frac{x}{32} -y)^{-2}})}{(\frac{1}{8} +
\frac{x}{16}- y)^2}\right], & \text{если} \ (x, y)\in\Pi_2; \\
0, & \text{если} \ (x, y)\in\partial_l(\Pi_2);\\
- A, & \text{если} \ (x, y)\in\partial_u(\Pi_2),
\end{cases}
\end{align*}
где $A=10^{-3}$, а также полином 3-й степени
$$
p(y)= -80y^3 + 88y^2 - 28y + 3.
$$
Обозначим через
$\Pi_3$ прямоугольник $[0, 1]\times (\frac{3}{16} +
\frac{x}{32}, \frac{1}{4}]$.



\begin{examnonum}
Определим отображение $F\in T^1([0, 1]^2)$, где
$$
F(x, y)=(x, g_x(y)).
$$
При этом семейство отображений в слоях зададим так, чтобы
отображение $g_0(y)\in C^1([0, 1])$ допускало $C^1\Omega$-взрыв
(здесь $C^1([0, 1])$ --- пространство $C^1$-гладких отображений
отрезка $[0, 1]$ в себя с $C^1$-нормой).
Положим
$$
g_x(y)=
\begin{cases}
16(x + 1)y(\frac{1}{4} - y), & \text{если} \
(x, y)\in [0, 1]\times [0, \frac{1}{8}); \\
\lambda_x(y), & \text{если} \ (x, y)\in\ov{\Pi}_1;\\
\mu_x(y), & \text{если} \ (x, y)\in\ov{\Pi}_2;\\
\frac{1}{4} + A\sin\frac{2\pi(8y - x)}{2 - x},
& \text{если} \ (x, y)\in \Pi_3;\\
p(y), & \text{если} \ (x, y)\in
[0, 1]\times (\frac{1}{4}, \frac{1}{2}];\\
4y(1 - y), & \text{если} \ (x, y)\in [0, 1]\times (\frac{1}{2}, 1].
\end{cases}
$$ 
\end{examnonum}

Заметим, что
$\{(0, ({g_0}_{ |[0, \frac{1}{8})})^{- n}
(\frac{1}{8}))\}_{n\geq 0}\subs\Omega(F)$ в то время, как
любая точка интервала $\{0\}\times(0, \frac{1}{4})$ является
блуждающей относительно слоя $\{0\}\times [0, 1]$.
Следовательно, в примере $\Omega_w(F)\ne\emptyset$.

Обратим внимание на то, что ни одна из точек множества $\Omega_w(F)$
отображения из примера не является $\omega$-предельной,
кроме того, здесь $\bigcup\limits_{(x, y)\in I}\omega_F((x, y))$ ---
незамкнутое множество.


\section{Доказательство теоремы 2}


В [4] доказано, что если множество $\Per(f)$ периодических
точек фактор-отображения $f$ отображения $F\in T^0(I)$ замкнуто, то
$C(F)=\ov{\Per(F)}$. Отсюда с использованием предложений 2 и 3
получаем, что в данном случае
\begin{equation}\label{e:4}
C(F)=\ov{\bigcup\limits_{x\in \Per(f)}
\{x\}\times C(\wt g_x)}=\ov{\bigcup\limits_{x\in \Per(f)}
\{x\}\times\Om(\wt g_{x|\Om(\Tilde g_x)})}.
\end{equation}
Таким образом, для доказательства теоремы 2 достаточно
убедиться в том, что $C(F)=\Om(F_{|\Om(F)})$.
Будем использовать $C$-функцию и
вспомогательные функции $\eta^*_n$.
Укажем, что в силу предложения 2 отображения
$\eta^*_n$ определены на множестве $\Per(f)$.
Воспользуемся последовательностью
натуральных чисел $j_n=2^{\frac{n(n+1)}{2}}$, $n\geq 1$.


\begin{lem}
Если множество $\Per(f)$ фактор-отображения $f$ отображения
$F\in T^0(I)$ замкнуто, то существует предел
$\Lim\limits_{n\to +\infty} \eta^*_{j_n|
{\Per(f, j_n)}}$ последовательности графиков функций $\eta^*_{j_n|
{\Per(f, j_n)}}$ в $I$, и справедливо равенство
$$
\ze^*=\Lim_{n\to +\infty}\eta^*_{j_n| {\Per(f, j_n)}},
$$
здесь $\ze^*$ --- график соответствующей функции в $I$.
\end{lem}

Действительно, т.\,к. при любом натуральном $n$ имеем
$C({\wt {g}_x}^n)=C(\wt g_x)$, то справедливо включение
$$
{\eta^*_{j_n}}_{| \Per(f, j_n)}\subseteq
{\eta^*_{j_{n+1}}}_{| \Per(f, j_{n+1})},\quad n\geq 1.
$$
Отсюда получаем, что последовательность графиков функций
$$
{\eta^*_{j_n}}_{| \Per(f, j_n)}=\bigcup\limits_{x\in
\Per(f, j_n)}\{x\}\times C(\wt g_x)
$$
сходится в $I$, и имеет место равенство
\begin{equation}\label{e:4.15}
\Lim_{n\to +\infty}{\eta^*_{j_n}}_{| \Per(f, j_n)}=
\ov {\bigcup\limits_{x\in \Per(f)}\{x\}\times C(\wt g_x)}.
\end{equation}
Из равенств (\ref{e:4}) и (\ref{e:4.15})
вытекает справедливость леммы 5.


\begin{lem}
Если множество $\Per(f)$ периодических
точек фактор-отображения $f$
отображения $F\in T^0(I)$ замкнуто, то
$\ze^*=\ze_{F_{|\Omega(F)}}$,
где $\ze^*$, $\ze_{F_{|\Omega(F)}}$ ---
графики соответствующих функций в $I$.
\end{lem}


{\bf Доказательство.} В силу равенства (\ref{e:4}) и
леммы 5 справедливо
\begin{equation}\label{e:4.17}
\ze^*\subseteq\ze_{F_{|\Omega(F)}}.
\end{equation}
Докажем противоположное включение
\begin{equation}\label{e:4.18}
\ze_{F_{|\Omega(F)}}\subseteq\ze^*.
\end{equation}
Будем использовать множество
$\Omega^*(F)=\bigcap\limits_{n=0}^{+\infty}F^n(\Omega(F))$.
Для доказательства (\ref{e:4.18}) установим включение
\begin{equation}\label{e:4.19}
\ze_{F_{|\Omega^*(F)}}\subseteq\ze^*.
\end{equation}
Предположим противное. Тогда найдется точка
$z^0(x^0, y^0)\in
\ze_{F_{|\Omega^*(F)}}\setminus
\ov{\bigcup\limits_{x\in \Per(f)}
\{x\}\times\Omega(\wt g_{x|\Omega(\Tilde g_x)})}$.
Возьмем произвольно и зафиксируем $\e>0$ так, чтобы не пересекались
$\e$-окрестности в $I$   \  $U_{\e}(z^0)$ и 
$U_{\e}(\ov{\bigcup\limits_{x\in \Per(f)}
\{x\}\times\Omega(\wt g_{x|\Omega(\Tilde g_x)})})$
точки $z^0$ и множества
$\ov{\bigcup\limits_{x\in \Per(f)}
\{x\}\times\Omega(\wt g_{x|\Omega(\Tilde g_x)})}$ соответственно.
Воспользуемся равномерной непрерывностью отображения $F$ и
по числу $\frac{\e}{2}> 0$ укажем $\de'> 0$ так, чтобы для любых
$x, x'\in I_1$, $y, y'\in I_2$, удовлетворяющих
неравенствам $|x - x'|<\de'$, $|y - y'|<\de'$,
выполнялось (\ref{e:4.14}).
Пусть для определенности (\ref{e:4.14}) справедливо лишь
для $\de'<\frac{\e}{2}$.
Так как $F:\Omega^*(F)\to\Omega^*(F)$ --- сюръекция, а
$z^0\in\Omega^*(F)$, то корректно определена отрицательная
полутраектория точки $z^0$ относительно отображения
$F_{|\Omega^{*}(F)}$.

Пусть точки
$\{z^{-n}\}_{n\geq 1}=\{(x^{-n}, y^{-n})\}_{n\geq 1}
\subset\Omega^*(F)$
образуют взятую произвольно и фиксированную
отрицательную полутраекторию точки
$z^0(x^0, y^0)$ относительно
$F_{|\Omega^*(F)}$, здесь
$f^n(x^{-n})=x^0$, $g_{x^{-n}, n}(y^{-n})=y^0$.
Тогда для любой точки $(x, y)\in \Omega^*(F)$ и любого $n\geq 1$,
удовлетворяющих неравенствам
$|x - x^{-n}|<\de'$, $|y - y^{-n}|<\de'$, верно
\begin{equation}\label{e:4.20}
F_{|\Omega^*(F)}^n(x, y)\in U_{\frac{\e}{2}}
(F_{|\Omega^*(F)}^n(x^{-n}, y^{-n}))
=U_{\frac{\e}{2}}(z^0).
\end{equation}
Так как для некоторого натурального числа $n_*$, не зависящего от точки
$z^0$, где $n_*=n_*(\bigcup\limits_{x\in \Orb(x^0)}\{x\}
\times U_{2, \de'}(\Omega(\wt g_{x|\Omega(\wt g_x)})))$,
имеет место неравенство
$\sum\limits_{n=0}^{+\infty}\chi_{A_6}(z^{-n})\leq n_*$
(здесь через $\Orb(x^0)$ обозначена периодическая орбита,
порожденная точкой $x^0$, $\chi_{A_6}$
--- характеристическая функция множества
$A_6=\Omega^*(F)\setminus
\Big(\bigcup\limits_{x\in \Orb(x^0)}\{x\}\times
U_{2, \de'} (\Omega(\wt g_{x|\Omega(\Tilde g_x)}))\Big)$,
$U_{2, \de'} (\Omega(\wt g_{x|\Omega(\Tilde g_x)}))$
--- $\de'$-окрестность множества
$\Omega(\wt g_{x|\Omega(\Tilde g_x)})$ в $I_2$),
то из (\ref{e:4.20}) следует, что найдется натуральное
число $n'_*\geq n_*$ такое, что при всех $n\geq n'_*$
для $n$-го полного прообраза относительно отображения
$F_{| \Omega^*(F)}$  \  $\frac{\e}{2}$-окрестности
$U_{\frac{\e}{2}}^{\Omega^*(F)}(z^0)=U_{\frac{\e}{2}}(z^0)
\bigcap\Omega^*(F)$ в $\Omega^*(F)$ точки $z^0$
справедливо
$$
(F_{| \Omega^*(F)})^{-n}
(U_{\frac{\e}{2}}^{\Omega^*(F)}(z^0))\subset\Omega^*(F)\bigcap U_{\e}
\Big(\ov{\bigcup\limits_{x\in \Per(f)}
\{x\}\times\Omega(\wt g_{x|\Omega(\Tilde g_x)})}\Big).
$$
В силу выбора окрестностей $U_{\e}(z^0)$ и 
$U_{\e}(\ov{\bigcup\limits_{x\in \Per(f)}
\{x\}\times\Omega(\wt g_{x|\Omega(\Tilde g_x)})})$
получаем
$$
U_{\frac{\e}{2}}^{\Omega^*(F)}(z^0)\bigcap
(F_{| \Omega^*(F)})^{-n}
(U_{\frac{\e}{2}}^{\Omega^*(F)}(z^0))=\emptyset
$$
при любом $n\geq n'_*$. Следовательно,
$z^0\notin\ze_{F_{|\Omega^*(F)}}$.
Полученное противоречие с выбором точки $z^0$
доказывает справедливость (\ref{e:4.19}).
Заметим, что $\ze_{F_{|\Omega^*(F)}}=\ze_{F_{|\Omega(F)}}$.
Действительно, т.\,к. график
$\ze_{F_{|\Omega(F)}}$ замкнут, то какова бы ни была точка
$z\in\Omega^*(F)\setminus\ze_{F_{|\Omega^*(F)}}$, справедливо
$z\in\ze_{F_{|\Omega(F)}}$ в том и только том случае, если
$z$ есть предельная точка $\ze_{F_{|\Omega(F)}}$.
Последнее невозможно, т.\,к. все точки множества
$\ze_F\setminus\Omega^*(F)$ блуждают относительно
графика $\ze_F$ (т.\,е. относительно множества $\Omega(F)$).
Таким образом, включение (\ref{e:4.18}) доказано.

Справедливость леммы 6 вытекает из (\ref{e:4.17}) и (\ref{e:4.18}).


Утверждение теоремы 2 следует из леммы 6 и равенства (\ref{e:4}).
Теорема 2 доказана.

В силу теоремы 2 глубина центра произвольного отображения
$F\in T^0(I)$ с замкнутым множеством периодических точек
в базе не превосходит 2.

Укажем также, что отображение из примера имеет непустое
множество
$$
C(F)\setminus \Big(\bigcup\limits_{x\in \Per(f)}\{x\}\times C(\wt g_x)\Big).
$$


\begin{thebibliography}{99}

\bibitem{1}
Аносов Д.В. {\em Об аддитивном функционально гомологическом уравнении,
связанном с эргодическим поворотом окружности} // Изв. АН СССР. --
Сер. матем. -- 1973. -- Т.\,37. -- \No\,6. -- С.\,1259--1274.

\bibitem{2}
Афраймович В.С., Быков В.В., Шильников Л.П. {\em О притягивающих
негрубых предельных множествах типа аттрактора Лоренца} //
Тр. Моск. матем. о-ва. -- 1982. -- Т.\,44. -- С.\,150--212.

\bibitem{3}
Beck C. {\em Chaotic cascade model for turbulent velocity
distribution} // Phys. Rev. -- 1994. -- V.\;E49. -- P.\,3641--3652.

\bibitem{4}
Ефремова Л.С. {\em О неблуждающем множестве и центре треугольных
отображений с замкнутым множеством периодических точек в базе} //
Динамич. системы и нелинейные явления. --
Киев: Ин-т матем. АН Украины. -- 1990. -- С.\,15--25.

\bibitem{5}
Ефремова Л.С. {\em О понятии $\Om$-функции косого произведения
отображений интервала} // Итоги науки и техн.
Сер. Современ. матем. и ее прилож. Тематич. обзоры:
Т.\,67. Тр. междунар. конф., посвященной 90-летию со дня рождения
Л.С.\,Понтрягина. Москва, 31 августа\,--\,6 сентября 1998~г.
Т.\,6: Динамические системы. -- М.: ВИНИТИ, 1999. -- С.\,129--160.

\bibitem{6}
Efremova L.S. {\em New set-valued functions in the theory of
skew products of interval maps} // Nonlinear Anal.
-- 2001. -- V.\,47. -- P.\,5297--5308.

\bibitem{7}
Шарковский А.Н. {\em О циклах и структуре непрерывного отображения} //
Укр. матем. журн. -- 1965. -- Т.\,17. -- \No\,3. -- С.\,104--111.

\bibitem{8}
Nitecky Z. {\em Maps of the interval with closed periodic set} //
Proc. Amer. Math. Soc. -- 1982. -- V.\,85. -- \No\,3. -- P.\,451--456.

\bibitem{9}
Шарковський О.М. {\em Неблукаючi точки та центр неперервного
вiдображення прямоi в себе} // Допов. АН УРСР. -- 1964. -- Т.\,7.
-- С.\,865--868.

\bibitem{10}
Федоренко В.В., Шарковский А.Н. {\em Непрерывные отображения интервала
с замкнутым множеством периодических точек} //
Исследов. дифференц. и дифференциально-разностных
уравнений. -- Киев: Ин-т матем. АН Украины. -- 1980. -- С.\,137--145.

\bibitem{11}
Куратовский К. {\em Топология}. Т.1. -- М.: Мир, 1966. -- 594 с.

\bibitem{12}
Coven E.M., Nitecki Z. {\em Nonwandering sets of the powers of maps
of the interval} // Ergod. Theory and Dynam. Syst. -- 1981.
-- V.\,1. -- P.\,9--31.

\bibitem{13}
Немыцкий В.В., Степанов В.В. {\em Качественная теория дифференциальных
уравнений}. -- М.--Л.: Гостехиздат, 1949. -- 550 с.

\bibitem{14}
Аносов Д.В. {\em Об одном классе инвариантных множеств гладких
динамических систем} // Тр. V междунар. конф. по нелинейным колебаниям.
Т.\,2. Качественные методы. -- Киев: Ин-т матем. АН Украины. -- 1970.
-- С.\,39--45.


\end{thebibliography}

\vskip 2cm

\postup{\it Нижегородский государственный}{\qquad \it Поступили}
\postup{\it университет}{\it первый вариант $09.09.2003$}
\postup{}{\it окончательный вариант $16.12.2005$}

\label{end}
\end{document}


